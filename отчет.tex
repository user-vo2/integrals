\documentclass[a4paper,12pt,titlepage,finall]{article}

\usepackage[T1,T2A]{fontenc}     % форматы шрифтов
\usepackage[utf8x]{inputenc}     % кодировка символов, используемая в данном файле
\usepackage{amsmath}
\usepackage[russian]{babel}      % пакет русификации
\usepackage{tikz}                % для создания иллюстраций
\usepackage{pgfplots}            % для вывода графиков функций
\usepackage{geometry}        % для настройки размера полей
\usepackage{indentfirst}         % для отступа в первом абзаце секции
 
\usetikzlibrary{graphs}
% выбираем размер листа А4, все поля ставим по 3см
\geometry{a4paper,left=30mm,top=30mm,bottom=30mm,right=30mm}

\setcounter{secnumdepth}{0}      % отключаем нумерацию секций

\usepgfplotslibrary{fillbetween} % для изображения областей на графиках

\begin{document}
% Титульный лист
\begin{titlepage}
    \begin{center}
    {\small \sc Московский государственный университет \\имени М.~В.~Ломоносова\\
    Факультет вычислительной математики и кибернетики\\}
    \vfill
    {\Large \sc Отчет по заданию №6}\\
    ~\\
    {\large \bf <<Сборка многомодульных программ. \\
    Вычисление корней уравнений и определенных интегралов.>>}\\ 
    ~\\
    {\large \bf Вариант 4 / 4 / 3}
    \end{center}
    \begin{flushright}
    \vfill {Выполнил:\\
    студент 102 группы\\
    Воробьев~Е.~Р.\\
    ~\\
    Преподаватель:\\
    Сенюкова~О.~В.}
    \end{flushright}
    \begin{center}
    \vfill
    {\small Москва\\2020}
    \end{center}
\end{titlepage}

% Автоматически генерируем оглавление на отдельной странице
\tableofcontents
\newpage

\section{Постановка задачи}

Решалась задача нахождения площади плоской фигуры, ограниченной тремя кривыми: $y = e^x+2; y = -\frac{1}{x}; y = -\frac{2(x+1)}{3}$ с точностью $\varepsilon$. Требовалось реализовать метод Симпсона, для нахождения непосредственно площади плоской фигуры, ограниченной тремя кривыми, с точностью $\varepsilon_2$. Для поиска вершин фигуры было необходимо использовать комбинированный метод (хорд и касательных) решения уравнений с точностью $\varepsilon_1$. Отрезок для применения метода нахождения корней должен был быть вычислен аналитически.\par

\newpage

\section{Математическое обоснование}

Из графика функции(рис.~\ref{plot1}) ясно, что вся фигура лежит в диапазоне $[-5; 0]$. Так как функция $f(x) = -\frac{1}{x}$ имеет разрыв в точке $x = 0$, то правая граница поиска пересечений функций равна $-0.1$. Корни всех уравнений лежат в диапазоне $[-5; -0.1]$, так как  
$$f_{1}(-5) = e^{-5}+2 \approx 2.006737 $$
$$f_{1}(-0.1) = e^{-0.1}+2 \approx 2.904837 $$
$$f_{2}(-5) = -\frac{1}{-5} = 0.2 $$
$$f_{2}(-0.1) = -\frac{1}{-0.1} = 10 $$
$$f_{3}(-5) = -\frac{2(-5+1)}{3} \approx 2.666667 $$
$$f_{3}(-0.1) = -\frac{2(-0.1+1)}{3} = 0.6 $$

\begin{equation}
 \begin{cases}
   f_{2}(-5) < f_{1}(-5), \\
   f_{2}(-0.1) > f_{1}(-0.1) 
 \end{cases}
\end{equation}

\begin{equation}
 \begin{cases}
   f_{1}(-5) < f_{3}(-5), \\
   f_{1}(-0.1) > f_{3}(-0.1) 
 \end{cases}
\end{equation}

\begin{equation}
 \begin{cases}
   f_{2}(-5) < f_{3}(-5), \\
   f_{2}(-0.1) > f_{3}(-0.1) 
 \end{cases}
\end{equation}

Таким образом, точки пересечения кривых нужно искать в диапазоне $[-5; -0.1]$.\par

Значение $\varepsilon_1$ выбрано равным $0.1$.Таким образом $\varepsilon_2 = \frac{\varepsilon}{\varepsilon_1} = 0.01$ 

\begin{figure}[h]
\centering
\begin{tikzpicture}
\begin{axis}[% grid=both,                % рисуем координатную сетку (если нужно)
             axis lines=middle,          % рисуем оси координат в привычном для математики месте
             restrict x to domain=-5:1,  % задаем диапазон значений переменной x
             restrict y to domain=-1:6,  % задаем диапазон значений функции y(x)
             axis equal,                 % требуем соблюдения пропорций по осям x и y
             enlargelimits,              % разрешаем при необходимости увеличивать диапазоны переменных
             legend cell align=left,     % задаем выравнивание в рамке обозначений
             scale=2]                    % задаем масштаб 2:1

% первая функция
% параметр samples отвечает за качество прорисовки
\addplot[green,samples=256,thick] {exp(x)+2)};
% описание первой функции
\addlegendentry{$y=e^x+2$}

% добавим немного пустого места между описанием первой и второй функций
%\addlegendimage{empty legend}\addlegendentry{}

% вторая функция
% здесь необходимо дополнительно ограничить диапазон значений переменной x
\addplot[blue,samples=256,thick] {-1/x};
\addlegendentry{$y=-\frac{1}{x}$}

% дополнительное пустое место не требуется, так как формулы имеют небольшой размер по высоте

% третья функция
\addplot[red,samples=256,thick] {-2*(x+1)/3};
\addlegendentry{$y=-\frac{2(x+1)}{3}$}
\end{axis}
\end{tikzpicture}
\caption{Плоская фигура, ограниченная графиками заданных уравнений}
\label{plot1}
\end{figure}

\newpage

\section{Результаты экспериментов}

Координаты точек пересечения (таблица~\ref{table1}) и площадь полученной фигуры.

\begin{table}[h]
\centering
\begin{tabular}{|c|c|c|}
\hline
Кривые & $x$ & $y$ \\
\hline
1 и 2 & -0.376567 & 2.686213 \\
2 и 3 & -1.822906 & 0.548575 \\
1 и 3 & -4.026760 & 2.017832 \\
\hline
\end{tabular}
\caption{Координаты точек пересечения}
\label{table1}
\end{table}


Результаты проиллюстрированы графиком (рис.~\ref{plot2}).

\begin{figure}[h]
\centering
\begin{tikzpicture}
\begin{axis}[% grid=both,                % рисуем координатную сетку (если нужно)
             axis lines=middle,          % рисуем оси координат в привычном для математики месте
             restrict x to domain=-5:1,  % задаем диапазон значений переменной x
             restrict y to domain=-1:6,  % задаем диапазон значений функции y(x)
             axis equal,                 % требуем соблюдения пропорций по осям x и y
             enlargelimits,              % разрешаем при необходимости увеличивать диапазоны переменных
             legend cell align=left,     % задаем выравнивание в рамке обозначений
             scale=2,                    % задаем масштаб 2:1
             xticklabels={,,},           % убираем нумерацию с оси x
             yticklabels={,,}]           % убираем нумерацию с оси y

% первая функция
% параметр samples отвечает за качество прорисовки
\addplot[green,samples=256,thick,name path=A] {exp(x)+2};
% описание первой функции
\addlegendentry{$y=e^x+2$}

% добавим немного пустого места между описанием первой и второй функций
%\addlegendimage{empty legend}\addlegendentry{}

% вторая функция
% здесь необходимо дополнительно ограничить диапазон значений переменной x
\addplot[blue,samples=256,thick,name path=B] {-1/x};
\addlegendentry{$y=-\frac{1}{x}$}

% дополнительное пустое место не требуется, так как формулы имеют небольшой размер по высоте

% третья функция
\addplot[red,samples=256,thick,name path=C] {-2*(x+1)/3};
\addlegendentry{$y=-\frac{2(x+1)}{3}$}

% закрашиваем фигуру
\addplot[blue!20,samples=256] fill between[of=A and B,soft clip={domain=-1.822906:-0.376567}];
\addplot[blue!20,samples=256] fill between[of=A and C,soft clip={domain=-4.026760:-1.822906}];
\addlegendentry{$S=3.563480$}

% Поскольку автоматическое вычисление точек пересечения кривых в TiKZ реализовать сложно,
% будем явно задавать координаты.
\addplot[dashed] coordinates { (-4.026760, 2.017832) (-4.026760, 0) };
\addplot[color=black] coordinates {(-4.026760, 0)} node [label={-90:{\small -4.026760}}]{};

\addplot[dashed] coordinates { (-1.822906, 0.548575) (-1.822906, 0) };
\addplot[color=black] coordinates {(-1.822906, 0)} node [label={-90:{\small -1.822906}}]{};

\addplot[dashed] coordinates { (-0.376567, 2.686213) (-0.376567, 0) };
\addplot[color=black] coordinates {(-0.376567, 0)} node [label={-90:{\small -0.376567}}]{};

\end{axis}
\end{tikzpicture}
\caption{Плоская фигура, ограниченная графиками заданных уравнений}
\label{plot2}
\end{figure}

\newpage

\section{Структура программы и спецификация функций}

\begin{itemize}
\item double f1(double x) \par
    Функция вычисляет и возвращает значение функции $f(x) = e^x+2$ в точке $x$.
\item double f2(double x) \par
    Функция вычисляет и возвращает значение функции $f(x) = -\frac{1}{x}$ в точке $x$.
\item double f3(double x) \par
    Функция вычисляет и возвращает значение функции $f(x) = -\frac{2*(x+1)}{3}$ в точке $x$.
\item double f1p(double x) \par
    Функция вычисляет и возвращает значение первой производной функции $f(x) = e^x+2$ в точке $x$.
\item double f2p(double x) \par
    Функция вычисляет и возвращает значение первой производной функции $f(x) = -\frac{1}{x}$ в точке $x$.
\item double f3p(double x) \par
    Функция вычисляет и возвращает значение первой производной функции $f(x) = -\frac{2*(x+1)}{3}$ в точке $x$.
\item double f1pp(double x) \par
    Функция вычисляет и возвращает значение второй производной функции $f(x) = e^x+2$ в точке $x$.
\item double f2pp(double x) \par
    Функция вычисляет и возвращает значение второй производной функции $f(x) = -\frac{1}{x}$ в точке $x$.
\item double f3pp(double x) \par
    Функция вычисляет и возвращает значение второй производной функции $f(x) = -\frac{2*(x+1)}{3}$ в точке $x$.
\item double root(double (*f)(double), double (*g)(double), double (*fp)(double), double (*gp)(double), double (*fpp)(double), double (*gpp)(double), double a, double b, double eps1) \par
    В данной функции реализован комбинированный метод нахождение корня уравнения $f(x) = g(x)$ с точностью $\varepsilon_1$ на отрезке [a ,b]. Возвращает точку пересечения функций $f(x)$ и $g(x)$.
\item double integral(double (*f)(double), double a, double b, double eps2) \par
    В данной функции реализован метод Симпсона нахождения $\int_a^b f(x)dx$ с точностью $\varepsilon_2$. Возвращает значение интеграла.
\end{itemize}

\begin{tikzpicture}
    \graph[nodes={align=center}, grow down sep, branch right sep] {
       {
            main -> "main.c" -> "main.o"  -> program;
            integral -> "main.c" -> "main.o"  -> program;
            root -> "main.c" -> "main.o"  -> program;
            f1 -> "functions.asm" -> "functions.o" -> program;
            f2 -> "functions.asm" -> "functions.o" -> program;
            f3 -> "functions.asm" -> "functions.o" -> program;
            f1p -> "functions.asm" -> "functions.o" -> program;
            f2p -> "functions.asm" -> "functions.o" -> program;
            f3p -> "functions.asm" -> "functions.o" -> program;
            f1pp -> "functions.asm" -> "functions.o" -> program;
            f2pp -> "functions.asm" -> "functions.o" -> program;
            f3pp -> "functions.asm" -> "functions.o" -> program;
       }
    };
\end{tikzpicture}

\newpage

\section{Сборка программы (Make-файл)}

\begin{verbatim}
all: program
program: main.o functions.o
    gcc -m32 -o program main.o functions.o -lm
main.o: main.c
    gcc -m32 -std=c99 -c main.c
functions.o: functions.asm
    nasm -f elf32 functions.asm
clear:
    rm *.o
\end{verbatim}

\begin{tikzpicture}
    \graph[nodes={align=center}, grow down sep, branch right sep] {
       {
            "main.c" -> "main.o" -> makefile -> program;
            "functions.asm" -> "functions.o"-> makefile -> program
       }
    };
\end{tikzpicture}

\newpage

\section{Отладка программы, тестирование функций}

Тестирование и отладка производилась при помощи опции командной строки  \textit{-test}.

\subsection{Тестирование функции вычисления корня}
\begin{table}[h]
\centering
\begin{tabular}{|c|c|c|c|c|c|c|}
    \hline
     тест & номера функций & левая граница & правая граница & точность & ответ \\
    \hline
     1 & 1 и 2 & -1.0 & -0.1 & 0.1 & -0.396115 \\
    \hline
     2 & 1 и 3 & -5.0 & -3.0 & 0.01 & -4.026753 \\
    \hline
     3 & 2 и 3 & -2.0 & -1.0 & 0.001 & -1.822878 \\
    \hline
\end{tabular}
\end{table}
\clearpage

\subsection{Тестирование функции вычисления интеграла}

\begin{table}[h]
\centering
\begin{tabular}{|c|c|c|c|c|c|c|}
    \hline
    тест & номер функции & нижняя граница & верхняя граница & точность & ответ \\
    \hline
     1 & 1 & -3.0 & 1.0 & 0.1  & 10.668496 \\
    \hline
     2 & 2 & -3.0 & -2.0 & 0.01  & 0.393999 \\
    \hline
     3 & 3 & -4.0 & -3.0 & 0.001  & 1.663556 \\
    \hline
\end{tabular}
\end{table}
\clearpage

\newpage

\section{Программа на Си и на Ассемблере}

Исходные тексты программы имеются в архиве, который приложен к этому отчету.

\newpage              

\section{Анализ допущенных ошибок}
В функцию root добавлено условие выхода за пределы отрезка с корнем. \par

В функциях f1, f1p и f1pp были исправлены ошибки, из-за которых функции давали неверный результат.\par


\newpage
\begin{raggedright}
\addcontentsline{toc}{section}{Список цитируемой литературы}
\begin{thebibliography}{99}
\bibitem{math} Ильин~В.\,А., Садовничий~В.\,А., Сендов~Бл.\,Х. Математический анализ. Т.\,1~---
    Москва: Наука, 1985.
\end{thebibliography}
\end{raggedright}
\end{document}
